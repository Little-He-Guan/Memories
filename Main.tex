\documentclass[UTF8,oneside]{ctexbook}

\begin{document}
\chapter{}
{\Large \emph{当}}我抬头往窗外看时,便注意到太阳的光芒比之前更加明亮了。一年前,当我还在高中教室里的时候,也是在这样看向窗外感受着愈发明亮的阳光,想着很快我将能够沐浴在其中。而教室里的表在这时,往往都指向了九点四十五分。

对于高中生活的记忆,总体来说,是我不愿意回忆的。我讨厌为了分数而拼命地刷题,讨厌被应付和敷衍的与高考无关的活动,讨厌那些被占用的自由时间(尤其是假期),讨厌那些因为事物具有甚至是潜在的影响成绩的可能就对它们的封禁,以及,讨厌让这些出现的人。高中的很多时光,便是在我想要抗争这些事物中度过的。人们说,``上帝在为你关上了一扇门的时候也会为你打开一扇窗,''而这些我的对抗,也像这句话说的一样,在我不知不觉中却又成为了我不愿忘记的东西。

所以当时间来到我的高三时期的时候,尽管耳边充斥着家长和老师不厌其烦的强调,我却并未在态度上有着多大的转变---我不想向高考妥协。不过高考也没有向我妥协。即使它本身还要等待快一年才会出现,但它的淫威却早已穿过了时间长河,就好像生长素那样,向在``胚芽鞘尖端下面''的我们输送着高考的影响。于是我身边的环境就这样改变了,即使只用人的肉眼,也能观察出很多的变化,比如课间外出的人曾经不算少,但现在却只有较少的一些。我在这其中算是一个十分奇怪的人,在以前外出的人更多的时候不怎么出去,可当人更少的时候,我却愈发喜欢出去了。

学校为了避免干扰或者是别的一些考虑,把高一高二的教学楼和高三的教学楼分开了。这两栋楼之间隔着的距离不是很短,但也不是很长。如果我能一下课就离开教室的话,走到另一边的教学楼再走回来,时间还是能够的。可是也就仅此而已了,正常下课的时间不能使我多在那边逗留一会,于是每天仅有一次的大课间便成为了我颇为珍惜的时光。我是大课间为数不多的去他们那边教学楼逛的高三学生之一。他们来我们这边的人数要更多一些,不过主要原因还是因为充值饭卡的地点在我们楼里。一到周末---充饭卡的阿姨可以下班休息但我们还要继续上课的日子,来我们这的高一高二学生也几乎没有了。这个问题曾经困扰了我一段时间。后来有一次我去高一高二的楼找曾经的老师,我刚想与他讲述曾经发生过的故事,但还未等我有机会说出,他就已经问道,``不学习跑到这里干嘛?'' 这时我便明白,在我们两栋楼之间已经隔了一层可悲的厚障壁了。

在高一高二的楼那里逛久了也没什么太好逛的。无论如何,那也是一幢教学楼,楼里很少会发生些娱乐活动(大概一年前,一场``开飞机''在楼道里进行了。不过在其中发生了一些意外,导致其中有个人当场就口吐白沫了。后来学校就颁布了禁令,禁止在楼道中的这些行为)。过了一段时间以后,我就开始只是默默地看着路上的人群。我之前买了一副墨镜,另外,为了模仿\emph{Gone With The Wind}里的Rhett Butler,我还买了一顶黑色的巴拿马帽。这两件装备平时给我带来了不少乐趣,但当我戴着他们去观察高一高二大课间外出的人群时,我反而觉得孤独。于是有一天我摘下了它们,脱掉了象征着我高三身份的校服,并且混入到了开心地外出的人群里。突然间我不再孤独了,我只是一个普通的高一高二学生,一个大课间想出来放松买点零食吃的学生,一个想出来到处看看风景的学生,甚至,听着其中夹杂的情侣们的聊天,我还可以假装我就是一对情侣中的一员,而我的女朋友正在和我聊天、调情。

在我们学校的广场中央,有着一个大喷泉。每天下午放学和晚自习开始之间的时间内,它都会定时喷水。在规章上,下午放学和晚自习开始隔了一个小时(后来好像成为了50分钟),不过那毕竟只是规章,年级主任们总会对它进行一些灵活地调整以保证晚自习能提前一段时间进行。这样,这个短暂的休息时间并只剩可怜的30-40分钟左右了。学生们没有足够的时间外出,而这灯光环绕下的大喷泉大概也是学校里最浪漫的地方之一了,这样每到晚上,这个地方便聚集了不少情侣。我并没有荣幸地加入到这些情侣中,也没有太多观赏喷泉的意愿,于是我也就很少在这个时候去到喷泉了。只是有一次,在听到班级同学谈论八卦的时候,偶然听到了这喷泉边发生的一件震动情侣圈的大事。

这个喷泉边的情侣们不光是我们学生知道,有各种消息来源的学校领导们自然也能知道。据说我们的年级主任一直想抓些``早恋典型'',不过迟迟没有抓到(大概是没有确凿的证据)。喷泉边的探照灯笔直地射向天空,这些光柱和周围的彩灯隐约地照亮了些周围的环境,映出了一个个人影。其中大多数影子都在四散而逃,像是在逃离灯光的照射一样。还有一个单独的影子,虽然都是黑灰色,但却很容易把这个乱舞的影子和别的人影区分开来。广场上的科技楼上,广播站放着震耳欲聋的歌声,在这协调的主旋律中,却又能模糊地听见主任的叫喊声,年轻男女的叫喊声,以及,笑声。昏暗的天空给这个充斥着声音和光影的场景蒙上了一层幕布,即使我是在听别人转述这个故事,却也能觉得就是在身临其境的看一场闹剧了。

主任最后究竟有没有抓到他想要的``典型''我并不清楚,我也懒得去看学校张贴的那些公告。但我唯一敢断言的是,经过这个事情情侣们间的感情一定更加牢固了。

\chapter{}
{\Large \emph{人}}们可以清楚地在高中的学生中感受到青春的气息。即使班主任这个绝缘体走入了教室的回路,青春的电势差也还是存在于那里。当绝缘体被移除,青春的电场便推动着电子在学生中游走,而这样产生的电场变化又激发出了磁场,向更远的地方传播着青春的波动。它传播的是这样广泛,每一个人身上都留下了它的影子。

我们学校给每个班级都放了一排柜子,让学生们能存放自己的物品。柜子的总数要比学生数量多上一些,所以每个班级总有几个空柜子。在我高二和高三班级的一些空柜子里,存放着一些奇怪的东西:消毒水,螺丝刀,瑞士军刀,磁铁,电池 ... 这些东西是金立持\footnote{化名}的。我之前并不了解金立持这个人,只是知道这个名字属于我们班的最后一名。学生们总是倾向于关注颜值高的或者是成绩好的人,金立持这两项都不满足,再加上他的行为举止有些怪异,也就很少有人平常和他接触。在高二的时候,金立持就经常在课间往高三的楼跑。起初发现这会事的时候我只是以为他在高三有哥哥或者姐姐,但后来一些人聚在走廊听一个同学讲不知道怎么得到的金立持的秘密,这时我才知道金立持喜欢高三的一个学姐。

关于金立持和那个学姐的议论也就持续了几天。之后的一段时间,金立持慢慢地从我的视野中淡出了。在我高二的最后一个月,高三的学生们高考完离校了。由于时间十分紧张,我往往要到学校的厕所来排出前一天吃下的废物。尽管厕所散发着尿素分解后的氨的刺鼻气味,但我还是喜欢在厕所里独处的时光,不仅是因为教室里的氛围让我感到压抑,更重要的是在这时我才真正感觉的我的时间,我的生命属于我自己。虽然只有短短的几分钟,但我却可以自由地让思维游荡。只有在这里我才能够暂时忘掉每天写在黑板上的作业,桌子上堆成小山的书,在走道发试卷的课代表,和那些突然出现监督学习,安排任务和催促我的老师。可是这样的机会又是很少的,大多数学生一到下课不是在教室睡觉就是跑到厕所解决生理需求了。而一旦有别人也在厕所里,我的好心情就被莫名其妙的破坏了。高三学生离开的这段时间是我唯一的能够几乎天天都有机会跑到高三的教学楼享受空无一人的厕所的一个月\footnote{我高二时新楼还未建好,高二和高三的楼隔的还不远。}。然而正是在这一个月的某一天,我在上厕所的途中看见了去高三楼的金立持,他的故事也开始在我面前展开。

在那次发现金立持后我就开始刻意在上厕所的路上关注他的身影,毕竟高三的学生都走了,他还去高三的楼干什么呢?于是我又在路上看到了他几次,直到有一次他也看见了我。我向他打了个招呼,正准备离开,不料他却直直地朝我走来。我扫了一眼他手中拿着的东西,大多是文具,但也有些像他放在柜子里那些事物一样奇怪的东西。我感觉有些尴尬,而这时他已经走到了我的面前。

``高三的学生都走了东西也不拿走,我自己捡了好多都盛不下了,要不那天咱一块去拾点?''

我被这猝不及防的邀请震住了。说实话我对捡文具什么的并不感兴趣,我自己的文具本来就是足够的。另外,我心里也并不是很赞同他的这个行为,假如有学生忘拿了东西又回来拿呢?然而我自己竟最终同意了这个邀请,大概是他的话勾起了我的兴趣。虽然我不打算拿走什么,但是去看看高三的学生都留下什么也算是学习生活中少有的有趣的活动了。

就这样,在我们即将放暑假的某个晚自习下课,我和金立持踏着月光进入了高三教学楼。在路上他语重心长地提醒我各种注意事项,比如要小心在楼道里巡逻的保安。他还开始给我讲他和保安周旋的故事,但讲得实在是不怎么生动有趣,于是我随口问道:

``你喜欢的那个女的怎么样了?''

他的故事戛然而止。我还以为接下来他要问我怎么知道的,可他最后叹了一口气,说:

``我打算今天去看看她的柜子。我之前去都没去她班,就想留着最后一次再看的... ''

我抬起头,在昏暗的光线下我分明看到他的嘴唇在颤抖。

来到目的地的时候我突然想上厕所了,于是金立持和我分别走入了教室和厕所的门。但当我也走进教室时,却发现金立持不见了。高三楼的灯自然是不会打开的,在这个环境下找一个人并不容易,再加上他之前提醒我要小心保安,我又不敢去大声呼叫,只能自己走近柜子,想看看什么东西,同时心中暗骂他不靠谱。

这个教室的窗户没有被关上。激烈的晚风穿过打开的窗户,使我打了一个寒战。尽管没有灯,但我能从声音中清楚地分辨出窗帘在风的作用下不停地抖动。眼睛看见月光照在一些杂乱的桌椅上,鼻子里则闻到一股因为未打扫而产生的灰尘气味。有那么一个瞬间我感觉我走入了一个荒废多年的大楼。我刚打开一个柜子,突然发现旁边关闭的柜门上出现了一道白色的反光。我下意识地转过头,看到走廊那边一道光柱从转弯另一边反射过来 --- 保安的手电筒。我下意识地想到教室里的小黑屋\footnote{我们学校每个教室里都有一个小屋用来放置卫生工具和别的一些东西。},可当我打开门的时候,我却差点惊呼出来。即使只有从小黑屋的窗户透进的微弱光芒,可金立持的脸依然惨白无比。如果不是他僵硬地抬起头看了我一眼,我都要以为他已经死了。

我小声跟他说保安来了,可他并没有什么反应。当我的注意力不再集中在保安身上时,发现金立持在盯着他手中拿着的相片。我问了一句什么相片,可他还是没有反应。我把头侧了过去,在黑暗的环境下隐约看见这相片上记录着一对年轻男女亲昵地搂在一起。

当高三开学前我看完班级成员的表时,我便意识到我又和金立持被分在一个班了。他走进教室的时候我看他和之前并没有什么变化,还是一样的头型,一样的装束,一样的身材。他的脾气也依然是古怪的。我记不清到底是班主任任命还是他自告奋勇,反正他当上了卫生委员。偶尔我会听见他举着几本书大喊,``这谁的书?'',又或者是他站在讲台上嚷``别动黑板了!''他的要求大致是合理的,然而他说话的语气和身体动作却经常使得人们不愿意听从他的话,有时也会在私下谈论起这些事。

毕竟这也是高三了,有很多东西够我去考虑了,我也不想分出些别的精力再关注一个没什么交集的人。于是时间又缓缓地推进了几个月。我再次注意到金立持的时候是他抱着几卷彩纸走进教室。我便凑上去问,

``你拿这些纸是干嘛的?''

``叠千纸鹤。''

他边说边拿出一卷纸,然后认真地叠了起来。我看着他叠了一个千纸鹤并放进瓶子里以后就离开了。

我本以为他只是多找了个爱好,可是接下来的一段时间他像着了魔一样一有时间就叠千纸鹤,甚至有一次,他就当着班主任的面上课时叠。可能是因为他还是倒数第一的原因,班主任平时不怎么管他平时的一些对别人来说就要管的行为。然而这次他就坐在第一排,边听班主任讲题边叠,也一点没有掩饰的意思。

``金立持,为什么$|PB|$就等于$|PC|$呢?''

班主任面带微笑,向他提了一个问题。他并没有刁难金立持,这个问题还是十分简单的,而金立持平时也是学数学的,所以他答上来了。可他并未在坐下之后停下手中的动作,还是继续叠着千纸鹤。班主任又讲了一会,随后还是微笑着走下讲台,边讲边把金立持桌子上放千纸鹤的瓶子一把拿走,放到讲台上。

``你别动!''

金立持突然站起来说。

出人意料的是,班主任并没有生气,只是说,

``上课不能干这种事情。下课再还给你。''

班上的同学除了金立持都笑了。

后来同学们去向金立持询问,原来他叠千纸鹤是想送给喜欢的人。不过这个喜欢的人到底还是不是之前的学姐我不知道,大家也没有再问。我就这样看着他瓶子里的千纸鹤一天天变满,然后,直到有一天,瓶子不见了。金立持说他把千纸鹤寄出去了。于是我和同学们便都去打听那个人怎么回复的他。他面带喜悦的说:

``她说,她说让我不要太关心她,要专注学习。她说她并不优秀,不值得我这样...''

我脸上的表情有点奇怪,我默默地退出了人群。

直到今日,我还是能在空间看见金立持发一些说说。这些说说大多是记载他的生活见闻,比如学校怎么样,他现在的城市怎么样... 而这样的说说又几乎总是以夸赞的口吻结束。我高中同学有次开玩笑说他的说说就像营销号一样。奇怪的是,现在当我看到这些说说的时候,我仿佛又看到了金立持和这些故事...

\chapter{}
{\Large \emph{重}}新回头读我所写的事情,我看到这可能会给读者一种感觉就是我的高中生活充满了值得记忆的事情。恰恰相反,和大多数人一样,大多数高中的时间都是索然无味的。即使有让我感兴趣的事情发生,但往往在我真正沉浸在其中之前它们就突然结束了。大多数时间我都是在学习,我也只能学习。学校里的一些制度本身就阻止了更有趣的事情发生。

在高一的课间里,我偶尔还能参与到同学们进行的独特的活动当中 --- 这些活动包含拿着贴满蛋白质女王\footnote{生物必修一的课本上讲到蛋白质的那一节有一个充满肌肉的女性示例。}的圣杯(某同学的水杯)找生物老师签名,或是让垃圾桶成为一位同学的归宿... 我享受这些活动的氛围。但到了高二,以前的同学们被分散到了各个班级,班级间不远的距离成为了阻碍这些活动发生的天堑。在最开始的时候我曾经尝试让这些活动也能发生在我之后的班级上,不过失败的很彻底。后来在课间我也只能和我的一些没有在课间睡觉的朋友聊聊感兴趣的话题,像游戏,八卦之类的。我很快就厌倦了这样的生活,而之后几次想让生活变得更有趣的尝试导致了我和班主任之间十分严重的冲突。

所以当时间带着我来到高三的时候,我觉得之后的生活一定会更加糟糕。毕竟高三是高中最紧张的一年 --- 高考像是掠食者一样在后面穷追不舍,学校也竭尽全力压榨时间。我能从这样的情况中期待什么呢?但当我现在回头往前看的时候,却发现高三的时光,不像还带有好奇和些许热情的高一,也不像让我逐渐心灰意冷的高二,给我留下了反而是高中三年最多的精彩的回忆。大概高三离现在的时间最近对这个现象有一定的责任。但我觉得,高三的我和之前的我有了某种内在的不同。人们说``人的潜力是无限的。'' 也许,在越来越大的压力和更多的失望下,我转变了观察生活的方式,并开始注意到隐藏在处处不如意下的美好。

我们的学校建在一座山上,所以在学校内常有很大的高度差。高三楼所在地大概是教学区地势最高的地方。从这里延申出了三条下坡的路:一条往北直通向北大门,而两条往东连接着高一高二的楼。在楼之间大概一半的地方,这两条路急切地交汇出了一个有喷泉的小广场。站在广场的边缘,往地势低的北面看,目光便从这里开始展开了一片修长的视野,在其中包含着喷泉沿着刻意修建的阶梯景观流下的水和两边的对称的绿树和草坪。这微型的瀑布在流了大概50米左右猛然停止,把剩下的空地留给一片我们曾经在这跑操和升国旗过的广场。广场的北面尽头便是直接面对着马路的北小门。学校旁的马路往往是比较忙碌的,即使不是上下学时段,来往的车辆也并不是十分稀疏的。

我曾站在那喷泉广场上观看从高一高二楼下课的同学,但这件事情做久了也就慢慢觉得单调。广场的西南两面则分别对应了死气沉沉的高三教学楼和学校背后的同样死气沉沉的山,这样我在广场上眺望的另外的唯一选择便只能是向北看了。随着水在阶梯上缓缓流下,随着广场上的旗帜慢慢地飘荡,又随着马路上的车辆纷纷驶过,我突然间有了一个熟悉的感觉,一个新的生活又要开始的感觉。之后的某一天,我突然突发奇想,站在这个广场边的栏杆旁,伸出双臂,向着远处拥抱 --- 我在模仿 Gatsby 拥抱着 Daisy 家的绿光。我承认这是一个相当肤浅的模仿。我并没有在像他一样向着自己的梦想伸手,事实上我在那个时候都不知道我的梦想到底是什么。但我却从这个肤浅的模仿中收获了不止一点慰籍。

可是一年的时间是相当漫长的,过了大概两个月的时间,我也再没有去那小广场的欲望了。随后寒假晚晚地到来,也因为疫情晚晚地结束。虽然我们的网课早就开始了,但是即使是在家上网课跟在学校相比也绝对能称得上是假期了。我们高三自然是疫情后第一批开学的学生,学校把我们暂时安排在了便于通风和保持社交距离的高一高二新教学楼。可是新楼的更加宽敞的教室和更加明亮的灯光却并未让我感到任何的放松,反而高考临近的压力和严厉防控措施下的限制使我觉得喘不过气。这段时间不得不说让我难以承受。每天闻着穿透口罩的消毒水味,提防着忘带早上中午都要家长签字的体温监测表,忍耐着完全只充斥着文化课的课表(虽然这时候的体育课大多数人也都是在教室自习,但至少大多数时候老师也会出面,我也能到操场享受下自由时光),再加上疫情给了学校一个优秀的管理学生的理由 --- 甚至只是因为为了解压,放学时去超市买点饮料喝被老师发现也会被批评,我觉得高中三年最黑暗的日子也就莫过于此了。

而高一高二一个月之后的返校又把我们赶回了曾经的老教室。在这个课间难以外出的特殊时段,我要寻找一些别的手段来把我从这地狱中拯救出来。一个偶然的念头使我回忆起了高一时的一件往事。在高一刚开始的某段时间里,我和另外两位男同学曾多次与一位女孩在放学的路上一起行走,聊聊天。这个女孩倒是给我留下了不浅的印象,不仅是因为我们一起走过放学的路,还因为在某次大雨我忘带伞的时候她帮我打伞。可惜到了高二这事情也慢慢地无疾而终了。听其中一位自己说追求过那女孩的男同学说,她高二有男朋友了,不过我并不觉得这是这件事结束的主要原因。无论如何,那也是过去的事情了。我对它也没有什么伤感,但它确实提醒了我,为什么在高三后期的上学放学,我就不能找个人在路上聊天呢?

于是在这最后两个月的时光,我开始注意起路上落单的女孩了。确实,如果只是为了聊天的话,性别倒是无所谓的。然而我还只是一个17岁的青少年啊,即使有着学业的压力,我体内的荷尔蒙也依然旺盛。如果能找一个女孩聊聊天,自然比找一个男孩聊天要有趣多了。可是这也使得这样做的难度增加了不少。同性之间也许不在意路上聊天这种行为,但很多异性却可能多想,尤其是在学生时代这种敏感时期。我也就只能在路上装作偶然遇见一个认识的女孩,再一起走完剩下的路。

这种偶遇的方式也同时给我带来了很多的可能性 --- 我不知道今天又会遇见谁 --- 这也同时让我有了希望。在这大量的可能性中,我遇见了一个让我昨天还能梦到她的女孩,正是因此我有了写下了这一章的动力。我们之前只是认识,但并不熟悉。听说那个时候她谈了个男朋友,不过最开始我也只是偶然遇见她和她聊聊天,所以我对这个情况没有怎么在意。

就这样连续着十几天,我享受着这比较充实的上学和放学时间。后来有一天上晚自习的路上,我在前方又发现了她的身影。我本来并不想追上去,可那天的晚风却并不这样想。我一定盯着她摆动的裙子和腿走了不短的一段时间,直到我发觉鼻尖萦绕着风送来的一股清香。我一时分不清这香味是来源于夏天开的牡丹花还是她。于是我加快了脚步,并在接近楼梯口的地方和她并驾齐驱。

都已经走到楼梯口了,剩下的路也就相当的短。我一共没说几句话就要和她分开了。就在我要开始质疑这样热的夏天一路跑过来只为这几句话是否值得时,她突然转过头给了我一个微笑。她那双注视着我的黑眼珠里一定有魔法 --- 突然间,对于我来说,这不再是一个炎热的夏天了。连周围充斥着的一些老师讲课、班主任训话、或是同学们晚读的声音也消失的无影无踪了。春天又回来了。那个无忧无虑的春天,那个百花飘香,微风轻抚的春天,在那个时候我体内青春的欲望还在熊熊燃烧的春天,伴随着她的这个微笑,又回到了我的面前。春天的到来也一并赶走了三年高中以来的种种压力,不快,痛苦,让它们在这一刻都消失了。我沉浸在这个美好的时刻无法自拔,直到她转过身去走了几步我才能够开口说出道别的话语。

在之后的一段时间里,我每天都卡着她平时的时间点,希望在上学和放学的路上能见到她。可在经历了几次和她路上的``偶遇''后,她突然地从我的视野里消失了,或者是偶尔能看见她的时候她也在和她的男朋友一起走。就当我几乎要怀疑她是不是意识到了什么而故意避开我的时候,我又重新找到她了。到了这个快高考的时候,晚自习时间又提前了。为了节省时间,她的家长晚上放学时开车带着菜过来,她在里面用餐完直接回去学习。所以我不顾家长的反对,坚持每晚都要自己骑着车回家吃饭再回来,只是为了回来的路上她刚好吃完饭出来我们又能正好遇见。遗憾的是,即使我再怎么样期待,她之后再未在那个晚自习上课前要分别的楼梯那给我一个那样的微笑了。

确实,到现在这个地步有的时候我都觉得我要爱上她了。但是我不能爱上她,不仅是因为她男朋友的原因,我也知道我们是很不同的两种人,我明白我并不会是她需要的那个人。于是在我自己意识的主动介入之下,我也就把我们之间的交往控制在了只是上学放学时相遇聊聊天的地步。当然,即使是这样,我们遇见的次数也有些太频繁了。我觉得肯定她,甚至她男朋友,都会察觉到一些事情。但出乎我意料的是,她除了偶尔问过怎么今天还能碰见你之外就并未有过关于这个问题的别的什么动作,也并未拒绝我每次的上前搭讪。我觉得这样的一个情况是很好的。我并没有挖她男朋友墙角的心思,我想要的只是暂时沉浸在这种和她聊天的时光中来化解我在高中最后时刻的庞大的负面情绪。虽然我担心她和她男朋友可能会这样以为,但直到最后也并未发生什么事情。甚至在高考考场,也许是上帝想要帮我一把,让我又遇见了她。我自然抓住了这个机会来请求她对我微笑最后一次,她也没什么犹豫的答应了。

我承认,我刚开始说的梦到她更大的作用还是用来吸引读者。我知道,我们的缘分大概已经终结了,再介入她的生活无论对她还是对我都不是一件好事。我想留下的,也只有这段宝贵的记忆。事实上,我大概好几个月没有再想起过她了。而这个梦,大概也只是对我不愿忘记的记忆的重现,在那里,我又一次和她在上学路上相遇。我说``好巧啊,我又遇见你了,''而她正微笑着向我竖起大拇指。
\end{document}